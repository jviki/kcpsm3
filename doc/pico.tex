% Author: Jan Viktorin
% License: GNU GPL
% Created: 2010-1-2

\documentclass[10pt,a4paper]{article}

\def\KCPSM#1{\texttt{KCPSM#1}}
\def\FPGA{\texttt{FPGA}}
\def\Xilinx{\textit{Xilinx}}
\def\WIN{\textit{MS Windows}}
\def\VHDL{\texttt{VHDL}}
\def\NOT{\textbf{NOT}}
\def\HEX{\texttt{HEX}}

\def\ins #1{\texttt{#1}}

\begin{document}
\author{Jan Viktorin}
\title{Alternative assembler for \textit{PicoBlaze} (\KCPSM3)}
\maketitle

\begin{abstract}
\KCPSM3 is a processor core for use in \FPGA\ from \Xilinx. \Xilinx\ offers it to download and
	synthetize for free from their website. They ship it with an assembler (\texttt{KCPSM3.EXE}).
	This program can be run on \WIN\ only, because no source code is provided. Those days there are
	available (as I know) two other implementations -- in \texttt{C++} and in \texttt{Java}. 
	But nothing in pure C (that's a pity, isn't it?).

	I decided that it could be nice to have an open source assembler for \KCPSM3 written in C.
	This document describes my implementation of this idea.
\end{abstract}

\paragraph{Project goals}
I was trying to make a small program available under a GPL license, that does translation of \KCPSM3 source code
	to a \HEX\footnote{Careful, it is \NOT\ the Intel \HEX\ file, the output is same as the original \texttt{KCPSM3.EXE}
	does.} file. It can also produce some debugging info in a listing file. 

The program is written in C99, that means that it is possible to compile it for nearly any platform with
	only a few changes. I'm going to use the program on PC and on an embedded device -- here it can pass the assembled 
	instructions directly to the \KCPSM3\ processor in a connected \FPGA. The program aims to use as few memory as possible
	(depends	on the source program). It assembles the whole program in one pass. If you define all constants at the beginning
	it doesn't need to store much temporary information during the runtime (only future jumps needs that).

\paragraph{Implementation for PC, porting issues}
The source code I provide is intended for use on PC. Feel free to modify it for your current needs (according to GNU GPL 
	license). There are some issues that must be considered then.
\begin{enumerate}
	\item The program uses system call \texttt{mmap} for reading the source code. On an embedded device this is
		useless, so there must be written a buffering strategy for that. If you need this change, see module \texttt{buffer.c}.
	\item If you want to assemble only eg. a line of code (inline assembling?), the current implementation can not do that 
		(directly), it always outputs all 1024 instructions (because of calling \texttt{output\_flush}). It is not difficult 
		to change this behaviour.
	\item If you have a device with the \FPGA\ directly connected, it is possible to send the assembled program directly there.
		To do this, change the module \texttt{output}.
	\item It is also possible to run more instances of the assembler in different threads without collisions (each threat uses its
		own data storage, there are no global variables), each assembler run is thread safe.
	\item I recommend to create a special header file for every different platform, that needs some wild changes (as it is done
		for PC -- \texttt{pc.h}).		
\end{enumerate}

\paragraph{Differences between the original and my program}
My implementation is \NOT\ fully compatible with original program 
	shipped with the processor core. The most important differences are mentioned here:
\begin{enumerate}
	\item Registers in the basic form \texttt{"sX"} are case sensitive (thus \texttt{"sA"} is recognized, but \texttt{"sa"} 
		or \texttt{"Sa"} or \texttt{"SA"} is not recognized). This is realy important difference, because the original
		program treats everything case insensitive but labels case sensitive. 
	\item The original program writes many files, eg. \VHDL\ and \texttt{Verilog} source codes with
		the assembled program. My program writes only the file with the result and a listing file (if requested).
		I follow the Linux way of programming, that one program should be small and do only the basic job.
		It is not difficult to write a script that converts output of my program to a \VHDL\ code that looks 
		the same as the \Xilinx's \texttt{KCPSM3.EXE} does (eg. \texttt{https://github.com/fcelda/hex2vhd}).
	\item My program supports some shotcuts\footnote{Must be compiled with preprocessor flag \texttt{-DSHORTCUTS\_EXTENSION}.} 
		similar (and compatible) to those from \texttt{pBlazIDE}. Namely: \ins{ADDC}, \ins{SUBC}, \ins{COMP}, \ins{IN}, \ins{OUT}, 
	   \ins{EINT}, \ins{DINT}, \ins{RET}, \ins{RETI}. For instructions with indirect access (\ins{IN}, \ins{OUT}, \ins{FETCH}, 
		\ins{STORE}) it is not necessary	to write paranthesis around the second register operand.
\end{enumerate}

\paragraph{Compilation}
To compile the assembler, simply use \textit{GNU make} tool:
\begin{verbatim}
  $ make
\end{verbatim}

If you want to activate the \texttt{pBlazIDE}-like extension, type this instead:
\begin{verbatim}
  $ CFLAGS=-DSHORTCUTS_EXTENSION make
\end{verbatim}

\paragraph{Usage of the assembler}
After you successfully compile the staff, you can start to use the program like this:
\begin{verbatim}
  $ ./pico -i program.psm -o program.hex -l program.l
\end{verbatim}
You can skip any of the options. When you skip \texttt{-i} parameter (the input file), program tries to read from the \texttt{stdin}
	(but this can fail because of using \texttt{mmap}). When you skip the \texttt{-o} option (output file), the assembled result
	is written to \texttt{stdout}. Skipping of \texttt{-l} option will disallow creation of the listing file.

	It is possible to use another two options: \texttt{-q} (quite mode -- no error messages) and \texttt{-h} (prints help).

\end{document}
